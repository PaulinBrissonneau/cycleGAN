\chapter{Ouverture}

Ainsi nous considérons que nous avons suffisamment avancé pour pouvoir nous attaquer à un problème complet. Nous avons donc eu besoin de trouver un sujet d'application qui correspondait à tous les membres de l'équipe. Nous avons pu tirer les exigences suivantes : il est tout d'abord impératif que le sujet n'ait pas déjà été résolu de manière satisfaisante.En effet nous souhaitions avoir une réelle utilité à travers ce projet. De plus, toujours en lien avec cette utilité, nous voulions que le problème à résoudre soit éthique, c'est à dire qu'il contribue à un monde meilleur. Les recherches ont été difficiles et ont longtemps été au point mort. Heureusement, un article écrit par les plus éminents spécialistes du machine learning \cite{rolnick_tackling_2019-1} a pu nous donner un embryon de piste. En effet, il se trouve qu'il est aujourd'hui difficile d'avoir des cartes précises des émissions de gaz à effet de serre. L'article propose alors d'augmenter considérablement la précision de telles cartes, et donc leur résolution, en couplant les données avec des cartes géographiques aussi précises que nécessaire. Nous avons ainsi pensé qu'un CycleGAN légèrement modifié pourrait peut être parvenir à produire des résultats convenables. Tout ceci n'est que supposition à l'heure où ces lignes sont écrites, mais nous avons bon espoir de mettre à l’œuvre toutes nos connaissances dans le but de répondre à cette problématique qui malheureusement devient de plus en plus préoccupante.
