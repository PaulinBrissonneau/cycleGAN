\chapter{Le cycleGAN}

\section{Présentation de la problématique}

Les cycleGAN sont des architectures de GAN qui permettent de répondre à une problématique bien spécifique : le \textbf{transfert de style non appairé}, que nous expliciterons.

Le transfert de style consiste à transformer des données d'un \textit{style à un autre}. Le terme de \textit{style} est à prendre au sens large et les données que l'on manipule peuvent être de natures diverses. Il peut s'agir par exemple de transformer des images de pommes en images d'orange, de transformer un paysage d'été en un paysage d'hiver, de transformer une musique classique en rock, ou encore de modifier l'expression les expressions faciales d'individus présents sur une image. Quelques exemples sont présentés sur la figure ??.

Le transfert de style peut s'effectuer entre plusieurs \textit{classes de styles}, mais nous allons ici nous concentrer dans le cas binaire où l'on considère deux styles. La problématique est donc de transformer des images d'un style à l'autre, et ceci dans les deux sens.

Le transfert de style (à deux classes), repose sur deux banques de données, que l'on notera A et B. Suivant les données auxquelles nous avons accès, il existe deux cas différents :
\begin{itemize}
  \item Dans le cas où nous connaissons un appairage entre les images de A et de B, le problème est un \textbf{transfert de style appairé}. Le but est donc d'apprendre et de généraliser le transfert d'une donnée de A à une donnée de B à partir d'exemples de paires déjà existantes.
  \textit{Par exemple, si A représente des bâtiments de jour, et B représente des bâtiments de nuit, il est possible de prendre la même photo de jour et de nuit. Ces deux photos constituent une paire dont chaque élément est d'un style différent.}
  \item Dans le cas où chaque élément de A n'a pas de lien direct avec un élément de B en particulier, le problème est un \textbf{transfert de style non appairé}. Le but n'est plus d'apprendre et de généraliser le transfert d'une donnée de A à une donnée de B à partir d'exemples de paires déjà existantes, mais d'apprendre le transfert entre le style de A et le style de B, sans avoir d'exemple d'une telle transformation. Il faut donc \textit{comprendre} à un niveau sémantique les style de A et B.
  \textit{Par exemple, si vous voulez transformer une image de votre chien en image de chat, vous ne pouvez pas obtenir une banque d'image de chiens déguisés en chats. Vous devez donc travailler avec d'une part des images de chiens (A), d'autre part des images de chats (B), sans pouvoir former de paires entre A et B.}
\end{itemize}

Ces deux types de transfert de style se traitent différemment. Pour le transfert de style appairé, une structure de GAN classique suffit puisque le discriminateur peut aisément comparer l'image générée avec l'image \textit{idéale}. Ce problème, que nous ne développerons pas ici, est traité et manière efficace par différents algorithmes, dont \textbf{Pix2Pix}. Le transfert de style non appairé ne permet pas la comparaison à l'image-cible puisqu'il n'existe pas de paires. \textbf{Il faut donc utiliser d'autres architectures, comme par exemple le cycleGAN.}


\section{Principe général du cycleGAN}


\section{Les fonctions de coûts}


\section{Implémentation et résultats}