\chapter*{Conclusion}
\addcontentsline{toc}{chapter}{Conclusion}


Tout au long du rapport, nous avons exploré des concepts et outils du \textit{machine learning}. Le découpage du rapport suit le déroulement temporel du projet tout au long de cette année, qui nous a emmené des bases du \textit{machine learning} avec le perceptron multicouches, jusqu’au cycleGAN, en passant par les réseaux à convolutions et les GAN. À chacune des étapes, nous avons pu comprendre en profondeur les outils que l'on manipule et nous avons pu apprendre à les expliquer. Chaque étape s'est soldée d'une implémentation fonctionnelle des algorithmes.\\
Nous avons pu arriver au bout du cheminement et implémenter un cycleGAN de qualité, capable d'être exécuté sur le mésocentre Moulon, mais encore perfectible sur quelques points. C'est un ingrédient indispensable pour la poursuite du projet, puisqu’il nous permettra maintenant de nous attaquer à un problème concret et novateur, que nous sommes - pour l'instant - encore en train définir. Nous pouvons entrer avec confiance dans la deuxième phase du projet grâce à tous ces outils nécessaires à sa réussite.