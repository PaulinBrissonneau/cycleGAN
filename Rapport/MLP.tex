\chapter{Multi-Layer Perceptron}

\section{Les neurones}

Les technologies qui vont êtres présentées dans la suite se basent toutes sur l'idée de neurone artificiel, aussi appelé neurone formel. McCulloh et Pitts le formalise en 1943 \cite{mcculloch_logical_1943}.
Un neurone formel est composé de deux parties.\begin{itemize}
\item La première consiste à faire la somme pondérée par des poids des valeurs d'entrée du neurone auxquelles on peut éventuellement ajouter un biais. Les poids sont propres au neurone et il y a un poids par entrée.
\item La deuxième partie du neurone est la fonction d'activation. Cette fonction va s'appliquer sur le résultat de la somme pondérée. On choisit quasi-exclusivement des fonctions non-linéaires pour deux raisons : briser la linéarité (car dans le cas contraire le réseau serait assimilable à une seule matrice) et obtenir un résultat d'une certaine forme (par exemple une probabilité entre 0 et 1).
\end{itemize}
McCulloh et Pitts dans leur première ébauche du neurone formel considérèrent des neurones au résultat binaire à l'aide d'une fonction d'activation de Heavyside. Un tel neurone formel est représenté par la figure \ref{neurone_exemple}.



\begin{figure}[!h]
\centering
\includegraphics[width=300pt,valign=t]{"images/MLP/neurone_exemple2"}
\caption{Exemple de neurone dont les poids sont les coefficents $w_1$ à $w_2$, et le biais est $w_0$. La fonction de Heavyside s'applique à la somme pour donner une sortie du neurone dans $\{0, 1\}$. Cette image provient d'un article de Benharir et al. \cite{benharir_approche_2014}.}
\label{neurone_exemple}
\end{figure}

Ainsi formellement en utilisant les notant $X = \begin{pmatrix} x_1\\ \vdots \\ x_n \end{pmatrix}$  les entrées du neurone, $W = \begin{pmatrix} w_1\\ \vdots \\ w_n \end{pmatrix}$ les poids correspondants et $\varphi$ la fonction d'activation, $b$ le biais, un neurone correspond à une fonction $N$ telle que : $$N(x_1,...,x_n,w_1,...,w_n) = \varphi(b+ \sum\limits_{i} x_i w_i) = \varphi(W^T  X+b)$$



\section{Le perceptron}
En 1958, Frank Rosenblatt utilise l'idée des neurones artificiels pour inventer le perceptron \cite{rosenblatt_perceptron_1958}. Son idée est d'utiliser les neurones pour reconnaître des formes simples dans des images. Cependant dans cette forme simple, il conserve les neurones tels que définis en 1943 avec la fonction de Heavyside et une seule couche de neurones (c'est-à-dire que les différents neurones ne sont pas reliés entre eux). Cela limite l'intérêt du perceptron : il ne peut apprendre que des \textit{pattern} linéairement séparables. Minsky démontre par exemple que le perceptron est incapable d'effectuer un XOR \cite{minsky_perceptrons_2017}. L'idée est d'ajuster les poids pour que la sortie du neurone soit 1 si et seulement si l'entrée est dans l'ensemble que l'on cherche à reconnaître. Le schéma d'un perceptron est donné sur la figure \ref{perceptron}.

\begin{figure}[!h]
\centering
\includegraphics[width=250pt,valign=t]{"images/MLP/perceptron"}
\caption{Exemple de perceptron à $n$ entrées et $p$ sorties. Il peut donc servir de classificateur à $p$ classes.}
\label{perceptron}
\end{figure}

\section{Perceptron multicouche}
Un \textbf{perceptron multicouche}, \textbf{mutliperceptron}, ou \textbf{MLP} (pour \textit{Multi-Layer Perceptron}) est composé de différentes couches: \begin{itemize}
\item La couche d'entrée qui correspond aux valeurs des données.
\item Une couche de sortie qui correspond aux valeurs renvoyées par le MLP.
\item Une ou plusieurs couches cachées (\textit{hidden layers}) qui sont des couches de neurones reliées entre elles. À chaque couche, les sorties correspondent aux entrées de la couche suivante.

\end{itemize}

Écrivons formellement ceci. On définit $N$ la sortie de la $i^e$ couche :
$$N^{(i)}  = \begin{pmatrix} 1\\ n_1^{(i)}\\ \vdots \\ n_{p_i}^{(i)} \end{pmatrix}$$  Le 1 permet la présence d'un biais.\\ 
Définissons également $W^{(i)}$ la matrice des poids de la couche i.

$$W^{(i)} = \begin{pmatrix} w_{1,1}^{(i)}& \hdots & w_{1, p_i}^{(i)} \\  \vdots & \ddots &  \vdots \\ w_{p_i1}^{(i)} & \hdots & w_{p_ip_i}^{(i)} \end{pmatrix}$$

Soit $\varphi_k^{(i)}$ la fonction d'activation de ce même neurone et $\Phi^{(i)}((x_1,...,x_{p_i})) \to (\varphi_1^{(i)}(x_1), ... \varphi^{(i)}_{p_i}(x_{p_i}))$.\\

On notera que si le réseau possède $L+1$ couches (de $0$ à $ L$), $N^{(0)} = X$ et $N^{(L)} = Y$. Dès lors, $$\forall i \in [1,L], N^{(i)} = \Phi^{(i)}(W^{(i)}N^{(i-1)}). $$





\subsection{Quelques fonctions d'activation}
Il existe de nombreuses fonctions d'activation avec chacune leurs avantages et leurs défauts. En voici quelques-unes.

\subsubsection{Sigmoïde}
Cette fonction permet d'avoir des valeurs de sortie entre 0 et 1 mais souffre d'un problème de \textit{gradient vanishing}, c'est-à-dire que proche de sa zone de saturation le gradient sera très vite trop faible. Cela portera défaut à l'apprentissage lors de l'application de la rétropropagation du gradient. La fonction introduit un hyperparamètre $\lambda$. Son allure est donnée par la figure \ref{sigmoide}.

\begin{displaymath}
\varphi_\lambda (x) = \frac{1}{1+e^{-\lambda x}}
\end{displaymath}

\begin{figure}[!h]
\centering
\includegraphics[width=150pt]{"images/MLP/sigmoide2"}
\caption{Fonction sigmoïde\\}
\label{sigmoide}
\end{figure}


\subsubsection{Tangente hyperbolique}
Cette fonction permet d'avoir des valeurs de sortie entre -1 et 1. Elle souffre du même problème de \textit{gradient vanishing} que la sigmoïde. Son allure est donnée par la figure \ref{tanh}.
\begin{displaymath}
\varphi (x) = tanh(x)
\end{displaymath}

\begin{figure}[!h]
\centering
\includegraphics[width=150pt]{"images/MLP/tanh2"}
\caption{Fonction tangente hyperbolique\\ }
\label{tanh}
\end{figure}


\subsubsection{Softmax}
Cette fonction s'applique sur un vecteur, elle est souvent utilisée en sortie car elle permet d'obtenir des sorties homogènes à des probabilités (entre 0 et 1 et dont la somme sur les sorties vaut 1).
\begin{displaymath}
\varphi_\lambda (X)_j = \frac{e^{-x_j}}{\sum\limits_{k}  e^{-x_k}}
\end{displaymath}

\subsubsection{ReLu}
La ReLu (pour \textit{Rectified Linear Unit}) définie ci-dessous permet d'éviter le \textit{gradient vanishing} puisqu'elle ne possède pas de zone de saturation. Son inconvénient réside dans sa partie de gradient nul. En effet, si sa pré-activation (la somme pondérée de ses entrées) est négative, l'effet du neurone sur les couches suivantes sera nul. Autrement, le gradient de l'erreur commise par ce neurone sera nul, il ne pourra donc pas mettre à jour ses poids et restera dans cette position pendant tout l'apprentissage. On dit que le neurone \textit{meurt}. L'allure de la ReLu est donnée par la figure \ref{ReLu}.

\begin{displaymath}
\varphi (x) = max(0,x)
\end{displaymath}

\begin{figure}[!h]
\centering
\includegraphics[width=150pt]{"images/MLP/reLu2"}
\caption{Fonction reLu}
\label{ReLu}
\end{figure}

\subsubsection{Et bien d'autres...}
Il existe beaucoup d'autres fonctions d'activation ayant divers effets. Ce champ de recherche est très actif : de nouvelles fonctions sont découvertes chaque année dont l'effet n'est pas toujours bien compris. On peut par exemple citer la fonction Mish (2019) \cite{misra_mish_2019-1} ou la softexponentielle(2016)  \cite{godfrey_continuum_2016}.

\begin{figure}[!h]
\centering
\includegraphics[width=150pt]{"images/MLP/mish2"}
\caption{Fonction Mish}
\label{Mish}
\end{figure}


\section{Rétropropagation}
\subsection{Principe}

On définit une fonction de coût $J$. Cette fonction mesure l'écart entre ce que renvoie notre MLP et ce qu'il devrait renvoyer. Notre objectif est donc de la minimiser afin d'obtenir la plus grande précision possible. Pour cela on va agir sur les seuls paramètres modifiables : les poids. On va ainsi calculer le gradient $\nabla J$ à l'aide des dérivées partielles par rapport aux poids de la dernière couche de neurones. Dès lors, par la règle de la chaîne, on peut calculer les dérivées partielles par rapport aux poids de la couche précédente et ainsi de suite. Ces dérivées partielles seront utilisées par les algorithmes d'optimisation pour minimiser $J$ en mettant à jour les poids par itérations successives. Le principe de la rétropropagation de l'erreur est illustré par la figure \ref{backward}.

\begin{figure}[!h]
\centering
\includegraphics[width=160pt]{"images/MLP/forward"}
\hspace*{5mm}
\includegraphics[width=160pt]{"images/MLP/backward"}
\caption{Principe de la rétropropagation du gradient. À gauche, la propagation avant, qui permet de calculer la sortie du réseau en fonction de l'entrée. À droite, la rétropagation de l'erreur, qui permet de calculer l'influence de chaque poids sur l'erreur en sortie. Ces deux schémas proviennent d'un article de Lillicrap et al. \cite{lillicrap_backpropagation_2020}}.
\label{backward}
\end{figure}

\subsection{Les fonctions de coûts}
Il en existe plusieurs, voici les principales.

\subsubsection{Mean Absolute Error}
Cette fonction compte les erreurs proportionnellement à leur importance.
\begin{displaymath}
MAE = \frac{1}{n} \sum\limits_{i = 0}^{n} |y_{th} - y_{mes}|
\end{displaymath}

où $y_{th}$ est la sortie théorique, et $y_{mes}$ est la sortie prédite par l'algorithme.

\subsubsection{Mean Squared Error}
Parfois, on est prêt à tolérer de faibles erreurs mais on refuse catégoriquement les aberrations. C'est exactement ce que fait l'erreur quadratique moyenne. Elle attribue un poids beaucoup plus important aux grandes déviations.
\begin{displaymath}
MSE = \frac{1}{n} \sum\limits_{i = 0}^{n} (y_{th} - y_{mes})^2
\end{displaymath}

où $y_{th}$ est la sortie théorique, et $y_{mes}$ est la sortie prédite par l'algorithme.

\subsubsection{Entropie croisée}
L'entropie croisée, introduite à l'origine pour mesurer la distance entre deux distributions de probabilités est très efficace pour les réseaux neuronaux dans le cas de la classification. En effet, dans ce cas il s'agit d'approximer une densité de probabilité.

\begin{displaymath}
Cross Entropy = - \sum\limits_{i = 0}^{n} y_{th}  log(y_{mes})
\end{displaymath}

où $y_{th}$ est la classe théorique, et $y_{mes}$ est la classe prédite par l'algorithme.

\subsection{Algorithmes d'optimisation}
La fonction de coût utilisée dans toute cette partie est $J = \sum\limits_{k} (y_{th} - y_{mes})^2$


\subsubsection{Descente de gradient stochastique}
Il s'agit de l'algorithme classique.  $w_{jk}^{(i)} \leftarrow w_{jk}^{(i)} - \alpha \frac{\partial J}{\partial w_{jk}^{(i)}}$ . Un coefficient $\alpha$ appelé \textbf{taux d'apprentissage} définit l'ampleur de la modification du poids. Plus $\alpha$ est grand, plus le pas est important. Ainsi un taux d'apprentissage trop faible entraînera une convergence trop lente ou un blocage dans un minimum local mais un taux trop élevé risque de ne jamais converger. Un exemple d'application de la descente de gradient stochastique est donné par la figure \ref{descentedegradientstochastique}.

\begin{figure}[!h]
\centering
\includegraphics[width=160pt]{"images/MLP/descentedegradientstochastique"}
\hspace*{5mm}
\includegraphics[width=160pt]{"images/MLP/descentedegradientstochastiquepropre"}
\caption{Exemple d'application de la descente de gradient stochastique. La figure représente le cheminement de la valeur des deux poids. La hauteur représente l'erreur. On voit bien ici que l'algorithme peut entraîner un blocage dans un minimum local.}
\label{descentedegradientstochastique}
\end{figure}



\subsubsection{Descente de gradient newtonienne}
Ici, on ajoute un terme d'inertie à la modification du poids cela entraîne l'apparition d'un nouvel hyperparamètre $\beta_1$ qui sert à doser cette inertie. On note $\nu$ le terme de descente au pas précédent : $$ w_j{k}^{(i)} \leftarrow w_{jk}^{(i)} + \beta_1 \nu - \frac{\partial J}{\partial w_{jk}^{(i)}}$$
Un exemple d'application de la descente de gradient stochastique est donné par la figure \ref{descentedegradientnewtonnienne}.
	
\begin{figure}[!h]
\centering
\includegraphics[width=160pt]{"images/MLP/descentedegradientnewtonnienne"}
\hspace*{5mm}
\includegraphics[width=160pt]{"images/MLP/descentedegradientnewtonniennepropre"}
\caption{Exemple d'application de la descente de gradient stochastique. La figure représente le cheminement de la valeur des deux poids. La hauteur représente l'erreur. On remarque que l'algorithme est bien plus instable mais explore plus et ne reste pas forcément bloqué dans les minimum locaux.}
\label{descentedegradientnewtonnienne}
\end{figure}

\subsubsection{RMSProp}
L'algorithme, dont un exemple d'application est donné sur la figure \ref{RMSProp}, est le suivant : on choisit les hyper-paramètres $\beta_2$, $\alpha$, et $\varepsilon$ tel que $\varepsilon$ soit petit mais non nul, on initialise $\nu$ à 0.
A chaque pas on calcule :

\begin{displaymath}
s \leftarrow \beta_2 s + (1-\beta_2)  (\frac{\partial J}{\partial w_{jk}^{(i)}})^2 \\
\end{displaymath}
\begin{displaymath}
w_{jk}^{(i)} \leftarrow w_{jk}^{(i)} - \frac{\alpha}{\sqrt{s+\varepsilon}}\frac{\partial J}{\partial w_{jk}^{(i)}}
\end{displaymath}

\begin{figure}[!h]
\centering
\includegraphics[width=160pt]{"images/MLP/RMSProp"}
\hspace*{5mm}
\includegraphics[width=160pt]{"images/MLP/RMSPropPropre"}
\caption{Exemple d'application de RMSProp. La figure représente le cheminement de la valeur des deux poids. La hauteur représente l'erreur. L'algorithme est sensible aux minima locaux mais peu aux grands gradients.}
\label{RMSProp}
\end{figure}

\subsubsection{ADAM}
Il s'agit d'une combinaison de RMSProp et de la descente de gradient newtonienne.
L'algorithme est le suivant : on choisit les hyperparamètre $\beta_1$, $\beta_2$,  $\alpha$ et $\varepsilon$ tel que epsilon soit petit mais non nul, on initialise $\nu$ et $s$ à 0.
A chaque pas on calcule :
\begin{displaymath}
\nu \leftarrow \beta_1 \nu + (1-\beta_1) \frac{\partial J}{\partial w_{jk}^{(i)}} \\
\end{displaymath}
\begin{displaymath}
s \leftarrow \beta_2 s + (1-\beta_2) (\frac{\partial J}{\partial w_{jk}^{(i)}})^2 \\
\end{displaymath}
\begin{displaymath}
w_{jk}^{(i)} \leftarrow w_{jk}^{(i)} - \frac{\alpha}{\sqrt{s+\varepsilon}}\frac{\partial J}{\partial w_{jk}^{(i)}}
\end{displaymath}
Un n exemple d'application est donné sur la figure \ref{ADAM}.

\begin{figure}[!h]
\centering
\includegraphics[width=160pt]{"images/MLP/adam"}
\hspace*{5mm}
\includegraphics[width=160pt]{"images/MLP/adampropre"}
\caption{Exemple d'application de ADAM. La figure représente le cheminement de la valeur des deux poids. La hauteur représente l'erreur. L'algorithme est peu sensible aux minima locaux et beaucoup plus stable que la méthode de descente de gradient newtonnienne.}
\label{ADAM}
\end{figure}



\subsubsection{Comparatif des algorithmes}

Pour comparer les différents algorithmes, nous pouvons représenter sur un seul dessin leurs convergences. Le résultat est présenté par la figure \ref{Comparatifoptimiseurs}.

\begin{figure}[!h]
\centering
\includegraphics[width=220pt]{"images/MLP/comparatifoptimiseurs"}
\caption{Comparaison des 4 algorithmes présentés. La figure représente le cheminement de la valeur des deux poids. La hauteur représente l'erreur. On observe que ADAM est clairement meilleur que les autres.}
\label{Comparatifoptimiseurs}
\end{figure}

\subsection{Les batchs}
Souvent, la dérivée $\frac{\partial J}{\partial w_{jk}^{(i)}}$ n'est pas calculée sur une seule entrée. Afin d'obtenir une estimation plus précise du gradient de $J$, on le moyenne sur $n$ entrées. On appelle cela la méthode par batchs. Il est bon de remarquer que le choix de $n$ n'est pas anodin. Il s'agit ici encore d'un hyper-paramètre à régler qui influe sur la convergence du réseau. L'influence de la taille des batchs sera étudiée plus loin.



\section{Implémentation et résultats}
\subsection{Avant-propos}
Nous avons étudié l'impact de différents hyperparamètres sur les résultats du MLP. Sauf mention contraire, le jeu de données utilisé est MNIST. MNIST est un ensemble d'images en niveau de gris de chiffres écrits à la main. Le MLP est alors utilisé en tant que classificateur. Il doit reconnaître, à partir du niveau de gris des pixels, quel est le chiffre dessiné. La tâche typique de l'algorithme est présentée par une interface simple sur la figure \ref{MNIST_exemple}.\\
Les résultats de précision sont calculés sur un jeu de données de test composé de données choisies aléatoirement avant l’entraînement et n'étant pas utilisées pour celui-ci. Afin de mesurer l'effet des hyperparamètres, on trace toujours les courbes avec tous les hyperparamètres égaux sauf celui qui varient.
Les MLP utilisés pour les mesures ont été réalisés \textit{from scratch}. La configuration à partir de laquelle on fait varier les hyper-paramètres a été obtenue par ajustement par essais-erreurs et ne prétend pas être optimale. Toutefois, les valeurs observées étaient de l'ordre de celles trouvées dans la littérature. Les données sont moyennées sur 50 lancers de l'algorithme et l'intervalle de confiance à 95\% est indiqué.

\begin{figure}[!h]
\centering
\includegraphics[width=200pt]{"images/MLP/MNIST_exemple"}
\caption{Exemple de résultat de classification avec le MLP }
\label{MNIST_exemple}
\end{figure}

\subsection{Effet du nombre de neurones par couche}
On s'intéresse ici à la largeur du réseau, c’est-à-dire-dire le nombre de neurones par couche. Les résultats sont présentés sur la figure \ref{MLP_largeur}. On y observe  à la fois que prendre trop peu de neurones ne permet pas une classification satisfaisante, et à la fois que prendre trop de neurones est contre productif. Il s'agit alors de trouver le juste milieu.

\begin{figure}[!h]
\centering
\includegraphics[width=200pt]{"images/MLP/MLP_largeur"}
\caption{Courbes de précision sur les données de test en fonction de l'avancement de l'apprentissage, en faisant varier le nombre de neurones par couche.\\ }
\label{MLP_largeur}
\end{figure}

\subsection{Effet du nombre de couches de neurones}
On s'intéresse ici à la profondeur du réseau. Les résultats sont présentés sur la figure \ref{MLP_profondeur}. On notera que le cas on l'a une seule couche correspond à un perceptron simple. On observe que le MLP est plus performant que le perceptron simple. Cependant augmenter la profondeur du réseau réseau réduit sa vitesse de convergence.

\begin{figure}[!h]
\centering
\includegraphics[width=200pt]{"images/MLP/MLP_profondeur"}
\caption{Courbes de précision sur les données de test en fonction de l'avancement de l'apprentissage, en faisant varier le nombre de couches du réseau.\\ }
\label{MLP_profondeur}
\end{figure}

\subsection{Taille des batchs}
Comme vu précédemment la taille des batchs influe grandement sur l'algorithme.

\subsubsection{Méthode stochastique}
Il s'agit du cas où le batch est de taille 1. On met à jour les poids pour chaque exemple parcouru. On observe alors sur la figure \ref{batch1} une forte instabilité liée au fait que le gradient peut fortement varier d'un exemple à l'autre.
\begin{figure}[!h]
\centering
\includegraphics[width=200pt]{"images/MLP/batch1"}
\caption{Courbes de précision sur les données de test en fonction de l'avancement de l'apprentissage avec \textit{batch\_size} = 1. Un epoch correspond au passage d'un batch dans l'algorithme. }
\label{batch1}
\end{figure}


\subsubsection{Méthode par batch}
On illustre ici avec de très grands batch. On moyenne le gradient dans cet exemple sur 2048 images. Sur la figure \ref{batch2}, on n'observe plus l'instabilité initiale. Cependant de grands batch augmentent la sensibilité aux minima locaux et mène parfois à une saturation de la fonction de coût. De plus, une passe correspond au passage d'un batch dans l'algorithme. Ainsi, avec de très grands batch, le calcul de chaque pas peut devenir très long. Souvent, une grande taille de batch est associée à un fort taux d'apprentissage (car on estime mieux le gradient).
\begin{figure}[!h]
\centering
\includegraphics[width=200pt]{"images/MLP/batch2048"}
\caption{Courbes de précision sur les données de test en fonction de l'avancement de l'apprentissage avec \textit{batch\_size} = 2048. Un epoch correspond au passage d'un batch dans l'algorithme. }
\label{batch2}
\end{figure}

\subsubsection{Méthode par mini-batch}
On choisit des batch de taille plus restreinte (ici 32 images). On réduit l'instabilité pour qu'elle ne gêne pas trop la convergence, de plus le coût de calcul reste raisonnable. Enfin on garde une légère instabilité qui rend l'algorithme plus résistant aux minima locaux. Le résultat est présenté sur la figure \ref{batch3}.

\begin{figure}[!h]
\centering
\includegraphics[width=200pt]{"images/MLP/batch32"}
\caption{Courbes de précision sur les données de test en fonction de l'avancement de l'apprentissage avec \textit{batch\_size} = 32. Un epoch correspond au passage d'un batch dans l'algorithme. }
\label{batch3}
\end{figure}


\subsection{Initialisation des poids}
Si l'on veut faire évoluer les poids du réseau, il faut bien les initialiser. Ainsi, on peut s'intéresser à l'effet de leur initialisation sur l'apprentissage.
On commence par s'intéresser à une initialisation par loi normale. La moyenne et l'écart-type sont alors deux hyper-paramètres. Grâce aux résultats présentés par la figure \ref{MLP_initnormale}, on remarque bien qu'il est important de choisir des valeurs raisonnables pour obtenir une convergence.

\begin{figure}[!h]
\centering
\includegraphics[width=200pt]{"images/MLP/MLP_initnormale"}
\caption{Courbes de précision sur les données de test en fonction de l'avancement de l'apprentissage avec initialisation des poids selon une loi normale.}
\label{MLP_initnormale}
\end{figure}

On s'intéresse ensuite au cas d'une initialisation selon une loi uniforme. Les bornes sont alors deux hyperparamètres. Ici encore, la figure \ref{MLP_inituni} montre qu'il est important de choisir des valeurs judicieuses.


\begin{figure}[!h]
\centering
\includegraphics[width=200pt]{"images/MLP/MLP_inituni"}
\caption{Courbes de précision sur les données de test en fonction de l'avancement de l'apprentissage avec initialisation des poids selon une loi uniforme.}
\label{MLP_inituni}
\end{figure}

On remarque d'ailleurs que dans le cas optimal la courbe d'apprentissage est similaire pour les deux lois.

\subsection{Taux d'apprentissage}
Le taux d'apprentissage est un hyper-paramètre très important, il va déterminer la vitesse d'apprentissage et conditionner sa stabilité. Un taux d'apprentissage trop important risque de provoquer des oscillations dans la descente de gradient, voire même de faire diverger l'apprentissage. Alors qu'un taux d'apprentissage trop faible donnera un apprentissage lent. L'influence du taux d'apprentissage est donnée par la figure \ref{taux_app}. On observe l'existence d'une valeur optimale pour le taux d'apprentissage, celle-ci dépend des autres hyper-paramètres. Les essais sont moyennés sur 50 essais, sur les premiers batchs seulement, ce qui explique que l'on n'atteint pas une bonne précision.

\begin{figure}[!h]
\centering
\includegraphics[width=320pt]{"images/MLP/taux_app"}
\caption{Influence du taux d'apprentissage sur la convergence. À gauche, la précision atteinte sur les premiers batchs en fonction du taux d'apprentissage. À droite, l'évolution temporelle du coût en fonction du taux d'apprentissage.}
\label{taux_app}
\end{figure}